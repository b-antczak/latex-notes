\documentclass{report}
\usepackage[margin=1in, paperwidth=8.5in, paperheight=11in]{geometry}
%Math packages%
\usepackage{amsmath}
\usepackage{amsthm}
%Spacing%
\usepackage{setspace}
\onehalfspacing
%Lecture number%
\newcommand{\lectureNum}{17}
%Variables - Date and Course%
\newcommand{\curDate}{March 7, 2017}
\newcommand{\course}{CS 241}
\newcommand{\instructor}{Gord Cormack}
%Defining the example tag%
%\theoremstyle{definition}%
\newtheorem{ex}{Example}[section]
%Setting counter given the lecture number%
\setcounter{chapter}{\lectureNum{}}
%Package to insert code%
\usepackage{listings}
\usepackage{courier}
\usepackage{xcolor}
\lstset { %
    tabsize=2,
    breaklines=true,
    language=C++,
    backgroundcolor=\color{blue!8}, % set backgroundcolor
    basicstyle=\footnotesize\ttfamily,% basic font setting
}
%Package for images%
\usepackage{graphicx}

\begin{document}
%Note title%
\begin{center}
\begin{Large}
\textsc{\course{} | Lecture \lectureNum{}}
\end{Large}
\end{center} 
\noindent \textit{Bartosz Antczak} \hfill
\textit{Instructor: \instructor{}} \hfill
\textit{\curDate{}}
\rule{\textwidth}{0.4pt}
% Actual Notes%
\subsubsection{What we learned so far}
A compiler is a program that takes a high-level language (which in this case is WLP4) and converts it to equivalent MIPS assembly language:
\begin{enumerate}
\item Compiler reads the WLP4 language using a \textbf{scanner}
\item Scanner produces tokens and passes them to the \textbf{parser}
\item Parser reads the program using L(1) and constructs a parse-tree
\end{enumerate}
Now we are focusing on the next step, which is \textbf{context-sensitive analysis}.
\section{Context-Sensitive Analysis}
\begin{center}
\textit{(Aside)}
\end{center} There is such thing as context-sensitive grammar, which is just like context-free grammar, but the rules are a little bit more general. They look like:
$$\alpha A \beta \rightarrow \alpha \gamma B \qquad \text{(expand } A \text{ to } \gamma \text{ only if preceded by } \alpha \text{ followed by } \beta)$$
\begin{center}
\textit{(End of the aside)}
\end{center}
Up to now, we made sure that what we were reading was syntactically correct. Now we're focused on the \textit{semantics} of the language. If a program is syntactically valid, we now consider:
\begin{itemize}
\item \textbf{variables and procedures}: they can be undeclared, declared before use, have multiple declarations
\item \textbf{types}: return value of procedures, parameter lists, operators
\item \textbf{scope}: scope of variables in and out of procedures
\end{itemize}
Consider the following WLP4 program:
\begin{lstlisting}
int wain(int a, int b) {
	int i = 10;
	int *p = NULL;
	
	i = i + i; // VALID
	i = i + p; // ERROR: assigning a pointer to an address
	i = p + i; // ERROR: same as above
	i = p + p; // ERROR: same as above
	i = p - p; // VALID
	p = i + i; // ERROR: assigning an integer to a pointer
	p = i + p; // VALID 
	p = p + p; // ERROR: same as above
	p = p - p; // ERROR: same as above
	p = q; // ERROR: q is undefined
	return foo; // ERROR: foo is undefined
}
\end{lstlisting}
Recall that a pointer is structured as $\alpha + i$, where $\alpha$ represents a starting address and $i$ is the offset.
\\ Observe that using a context-free grammar would not work. We must use context-sensitive analysis instead. We'll use it to solve these issues:
\subsection{Variable Declaration Issues}
We use a \textit{symbol table}, similar to one that we used for MIPS. We'll track each variable's \textit{name}, \textit{location}, and \textit{type}. We'll construct our symbol table and access a particular item's symbol table with \texttt{symboltable[s]}. Referring to the CFG rules of WLP4 for variables:
\begin{center}
\begin{tabular}{ c | c }
\hline \textbf{Rule} & \textbf{Output for symbol table} \\\hline
dcls $\rightarrow \varepsilon$ & the symbol table is empty for dcls \\\hline
dcls$_0$ $\rightarrow$ dcls$_1$ dcl & merge the symbol tables of dcls$_1$ and dcl \\\hline
dcls$_0 \rightarrow$ type ID & store the type and ID into the symbol table for dcls$_0$
\end{tabular}
\end{center}
After we construct our symbol table, we'll evaluate the type of all expressions using \texttt{typeof[expr]}
\begin{center}
\begin{tabular}{ c | c }
Rule & input for typeof \\\hline
expr $\rightarrow$ term & typeof[expr] = typeof[term] \\
expr$_0 \rightarrow$ expr$_1$ + term &
\begin{tabular}{@{}c@{}}\texttt{typeof[expr$_1$] == typeof[term] == int}; return int \\ \texttt{typeof[expr$_1$] == int \&\& typeof[term] == intptr)}; return intptr \\ \texttt{typeof[expr1] == intptr \&\& typeof[term] == int}; return intptr \\ else, return invalid\end{tabular}
\end{tabular}
\end{center}

%END%
\end{document}