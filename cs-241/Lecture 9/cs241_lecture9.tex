\documentclass{report}
\usepackage[margin=1in, paperwidth=8.5in, paperheight=11in]{geometry}
%Math packages%
\usepackage{amsmath}
\usepackage{amsthm}
%Spacing%
\usepackage{setspace}
\onehalfspacing
%Lecture number%
\newcommand{\lectureNum}{9}
%Variables - Date and Course%
\newcommand{\curDate}{January 31, 2017}
\newcommand{\course}{CS 241}
\newcommand{\instructor}{Kevin Lanctot}
%Defining the example tag%
%\theoremstyle{definition}%
\newtheorem{ex}{Example}[section]
%Setting counter given the lecture number%
\setcounter{chapter}{\lectureNum{}}
%Package to insert code%
\usepackage{listings}
\usepackage{courier}
\usepackage{xcolor}
\lstset { %
    tabsize=2,
    breaklines=true,
    language=C++,
    backgroundcolor=\color{blue!8}, % set backgroundcolor
    basicstyle=\footnotesize\ttfamily,% basic font setting
}
%Package for images%
\usepackage{graphicx}

\begin{document}
%Note title%
\begin{center}
\begin{Large}
\textsc{\course{} | Lecture \lectureNum{}}
\end{Large}
\end{center} 
\noindent \textit{Bartosz Antczak} \hfill
\textit{Instructor: \instructor{}} \hfill
\textit{\curDate{}}
\rule{\textwidth}{0.4pt}
% Actual Notes%
\section{Regular Languages}
This topic focuses on building a \textit{compiler}, which takes high-level code and translates it into assembly language (rather than what we were doing up to this point, which was translating assembly language into machine code).\\
The steps in compiling a program from a high level language to an assembly language program are:
\begin{enumerate}
\item \textbf{Scanning:} create a token sequence
\item \textbf{Syntax analysis:} create a \textit{parse tree} (rather than a list of tokens, this is new)
\item \textbf{Semantic analysis:} create a symbol table and \textit{type checking} (which is new)
\item \textbf{Code generation:} similar, but more complicated for a compiler
\end{enumerate}
The goal of each of these steps is to find increasingly more sophisticated errors in a program. If there are no errors, then a successful compilation occurs; otherwise, print an error message.\\
The \textbf{Chomsky Hierarchy} can be used to find such errors.\\
We'll be compiling our own defined langauge, called WLP4 (CS 241's Waterloo Language Plus Pointers Plus Procedures).
\subsection{Approach to WLP4}
Use \textbf{regular expressions} to describe our language. A regular expression is a precise way of describing a set of strings (think of programs as a sequence of characters, which is actually what they are). We also define a \textbf{string} as a finite sequence of characters over some alphabet (our alphabet in this course will be some subset of the ASCII characters).
\subsubsection{Three Operations for Building up Languages}
\begin{enumerate}
\item \textbf{Union:}\\
$R \cup S$ is the union of set $R$ and $S$. If $R$ and $S$ are regular languages, then so is $R \cup S$. For example, if $R = $\{cow, pig\} and $S = $\{dog, cat\}, then $R \cup S = $\{dog, cat, cow, pig\}.
\item \textbf{Concatenation:}\\
Defined as $RS = \{\alpha\beta \, : \, \alpha \in R \wedge \beta \in S\}$ (i.e., take a word from $R$ and combine it with a word from $S$). If $R = $\{grey, blue\} and $S = $\{whale\}, then $RS = $\{greywhale, bluewhale\}.\\
Concatenation with the empty string, $\epsilon$, does nothing (i.e., $\epsilon\alpha = \alpha$).
\item \textbf{Repetition:}\\
If $R$ is a language, we can talk about $R^0, R^1, R^2, R^3, $ etc. To define every possible sequence of characters, we denote $R^*$ (a.k.a. a \textit{Kleene star}). For example, if $R = \{0,1\}$, then
\begin{itemize}
\item $R^0 = \{\varepsilon\}$
\item $R^1 = \{0,1\}$ (all single elements)
\item $R^2 = \{00, 01, 10, 11\}$ (all pairs of elements)
\end{itemize}
\end{enumerate}
\subsection{Two Types of Languages}
A \textbf{language} is defined as a set of finite sequences of characters from a defined alphabet. We can define a \textbf{finite} or \textbf{infinite} language. Infinite languages are denoted with a Kleene star, which means (as mentioned) that language contains every possible sequence of characters. An example of an infinite language is:
$$a* = \{\varepsilon , a, aa, aaa, ...\}$$
%END%
\end{document}