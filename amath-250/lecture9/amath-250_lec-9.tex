\documentclass{report}
\usepackage[margin=1in, paperwidth=8.5in, paperheight=11in]{geometry}
%Math packages%
\usepackage{amsmath}
\usepackage{amsthm}
\usepackage{amssymb}
%Spacing%
\usepackage{setspace}
\onehalfspacing
%Lecture number%
\newcommand{\lectureNum}{9}
%Variables - Date and Course%
\newcommand{\curDate}{May 22, 2017}
\newcommand{\course}{AMATH 250}
\newcommand{\instructor}{Zoran Miskovic}
%Defining the example tag%
%\theoremstyle{definition}%
\newtheorem{ex}{Example}[section]
%Setting counter given the lecture number%
\setcounter{chapter}{\lectureNum{}}
\usepackage{graphicx}
\usepackage{calc}
\usepackage{listings}

\lstset{frame=tb,
  aboveskip=3mm,
  belowskip=3mm,
  showstringspaces=false,
  columns=flexible,
  basicstyle={\small\ttfamily},
  numbers=none,
  breakatwhitespace=true,
  tabsize=3
}


\begin{document}
%Note title%
\begin{center}
\begin{Large}
\textsc{\course{} | Lecture \lectureNum{}}
\end{Large}
\end{center} 
\noindent \textit{Bartosz Antczak} \hfill
\textit{Instructor: \instructor{}} \hfill
\textit{\curDate{}}
\rule{\textwidth}{0.4pt}
% Actual Notes%
\subsubsection{Last Time}
Dimensional analysis, baseball thrown up in air example. We saw that
\begin{equation}
h = -\frac{1}{2}gt^2 + v_0t
\end{equation}
\section{Continuing baseball example}
Define characteristic time $t_c$ with $[t_c] = T$
$$t_c = \frac{v_0}{g}$$
Also define characteristic length $\ell_c$ with $[\ell_c] = L$
$$\ell_c = \frac{v_0^2}{g}$$
Define dimensionless time $\tau$ with $[\tau] = 1$
$$\tau = \frac{t}{t_c}$$
Define dimensionless height $H$
$$H = \frac{h}{\ell_c}$$
Insert $h = H\ell_c$ and $t = \tau t_c$ into (9.1)
\begin{align}
H\ell_c &= -\frac{1}{2}gt^2\tau^2 + v_0t_c\tau \\
H\ell_c &= -\frac{1}{2}\ell_c\tau^2 + \ell_c\tau \\
H\ell_c &= -\frac{1}{2}\ell_c\tau^2 + \ell_c\tau \\
H &= -\frac{1}{2}\tau^2 + \tau \qed
\end{align}
\section{Converting DEs to dimensionless form}
\begin{ex}
Mixing tank with constant flow rates $f_{in} = f_{out} = f$, which also means that the volume $V$ is also constant.
\end{ex}
\noindent
The DE for mass of salt $m(t)$ is
\begin{equation}
\frac{dm}{dt} = -\frac{f}{V}m + fc_{in}
\end{equation}
By inspection (will be made more formal in later lectures), we define:
\begin{itemize}
\item characteristic time: $t_c = \frac{V}{f}$
\item dimensionless time: $\tau = \frac{t}{t_c}$
\end{itemize}
Using the Chain Rule, we let $m = m(\tau) = m(\tau(t))$,
\begin{align}
\frac{dm}{dt} = \frac{dm}{d\tau}\cdot\frac{d\tau}{dt} = \frac{1}{t_c}\cdot \frac{dm}{dt}
\end{align}
Sub (9.7) into (9.6)
\begin{align}
\frac{1}{t_c}\frac{dm}{d\tau} &= -\frac{1}{t_c}m + fc_{in} \\
\frac{dm}{d\tau} + m &= t_cfc_{in} \\
\frac{dm}{d\tau} + m &= Vc_{in}
\end{align}
If $c_{in}$ is constant, then we can define characteristic mass $m_c = Vc_{in}$ with $[m_c] = M$. Now we define dimensionless mass
\begin{equation}
\mathcal{M}(\tau) = \frac{m(t)}{m_c} = \frac{m(\tau)}{m_c}
\end{equation}
We now solve
\begin{align}
\frac{d\mathcal{M}}{d\tau} &= \frac{d\mathcal{M}}{dm}\cdot \frac{dm}{d\tau} \\
&= \frac{1}{m_c}\cdot \frac{dm}{d\tau}
\end{align}
Sub (9.13) into (9.10)
\begin{align}
m_c\frac{d\mathcal{M}}{d\tau} + m_c\mathcal{M} &= m_c \\
\frac{d\mathcal{M}}{d\tau} + \mathcal{M} &= 1
\end{align}
The general solution is
$$\mathcal{M}(\tau) = 1 + Ce^{-\tau}$$
This DE defines all mixing tank problems, it's the simplest possible DE. We can convert the general DE to a particular solution by letting $\tau = \frac{t}{t_c}$ and $\mathcal{M} = \frac{m}{m_c}$
\begin{align*}
m(t) &= m_c\left[1 + Ce^{-\frac{t}{t_c}}\right] \\
&= Vc_{in} + De^{-\frac{f}{V}t} \qed
\end{align*}
\begin{ex}
Skydiver problem
\end{ex}
As shown in previous lectures, we have the following DE
\begin{equation}
m\frac{dv}{dt} = mg - \alpha v
\end{equation}
We have $[\alpha] = \frac{M}{T}$. By inspection (will formalize later), we define
\begin{itemize}
\item characteristic velocity: $v_{term} = v_c = \frac{mg}{\alpha} = t_cg$
\item characteristic time: $t_c = \frac{m}{\alpha}$
\end{itemize}
Convert (9.16) in ``one shot". Define
\begin{itemize}
\item dimensionless time: $\tau = \frac{t}{t_c}$
\item dimensionless velocity: $V = \frac{v}{v_c}$
\end{itemize}
Consider $v = v(\tau)$ and so $V = V(\tau)$. By chain rule
\begin{align}
\frac{dv}{dt} &= \frac{dv}{dV} \cdot \frac{dV}{d\tau}\cdot\frac{d\tau}{dt} \\
&= v_c\cdot \frac{dV}{d\tau}\cdot\frac{1}{t_c} \\
&= \frac{v_c}{t_c} \cdot \frac{dV}{d\tau}
\end{align}
Sub (9.19) into (9.16)
\begin{align}
m\cdot \frac{v_c}{t_c}\cdot \frac{dV}{d\tau} &= mg - \alpha v_cV \\
m\cdot \frac{mg}{\alpha}\cdot \frac{\alpha}{m}\cdot \frac{dV}{d\tau} &= mg - mgV \\
\frac{dV}{d\tau} + V &= 1
\end{align}
The result is exactly the same as the mixing tank problem!
%END%
\end{document}