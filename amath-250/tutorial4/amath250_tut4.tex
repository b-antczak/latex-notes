\documentclass{report}
\usepackage[margin=1in, paperwidth=8.5in, paperheight=11in]{geometry}
%Math packages%
\usepackage{amsmath}
\usepackage{amsthm}
\usepackage{amssymb}
%Spacing%
\usepackage{setspace}
\onehalfspacing
%Lecture number%
\newcommand{\lectureNum}{4}
%Variables - Date and Course%
\newcommand{\curDate}{June 1, 2018}
\newcommand{\course}{AMATH 250}
\newcommand{\instructor}{}
%Defining the example tag%
%\theoremstyle{definition}%
\newtheorem{ex}{Example}[section]
%Setting counter given the lecture number%
\setcounter{chapter}{\lectureNum{}}
\usepackage{graphicx}
\usepackage{calc}
\usepackage{listings}

\lstset{frame=tb,
  aboveskip=3mm,
  belowskip=3mm,
  showstringspaces=false,
  columns=flexible,
  basicstyle={\small\ttfamily},
  numbers=none,
  breakatwhitespace=true,
  tabsize=3
}


\begin{document}
%Note title%
\begin{center}
\begin{Large}
\textsc{\course{} | Tutorial \lectureNum{}}
\end{Large}
\end{center} 
\noindent \textit{Bartosz Antczak} \hfill
\textit{\curDate{}}
\rule{\textwidth}{0.4pt}
% Actual Notes%
\section*{Problem 1}
An object is thrown up in the air and it falls back down. We define all of the usual parameters with the addition of $t_r$ which represents the time it takes the ball to come back to the ground after it was thrown.
\begin{enumerate}
\item[(a)] What can we say about $t_r$ in terms of $m,g,v_0$ using $B\Pi T$?

\vspace{0.5em}\noindent
Since we have $N = 4$, $r=3$, there is one dimensionless variable. Let's define 
$$\Pi = \frac{g}{v_0}t_r$$
By $B\Pi T$, $c = \Pi = \frac{gt_r}{v_0}$, which means that
$$t_r = \frac{cv_0}{g} \qquad \qed$$
\item[(b)] Solve the DE and find $t_r$

\vspace{0.5em}\noindent
By Newton's 2nd law, $F = ma \implies a = -g$.
\begin{align*}
\frac{dv}{dt} &= -g \\
v &= -gt + c_1
\end{align*}
Using $v(0) = v_0$, we have $c_1 = v_0$
\begin{align*}
\frac{dh}{dt} &= -gt + v_0 \\
h &= -\frac{1}{2}gt^2 + v_0t + c_2
\end{align*}
Applying $h(0) = 0$, we have $c_2 = 0$, and so
$$h(t) = -\frac{1}{2}gt^2 + v_0t$$
Setting $h(t) = 0$, we solve for $t_r$, which results in
$$t_r = \frac{2v_0}{g} \qquad \qed$$
\item[(c)] Consider a drag force now

\vspace{0.5em}\noindent
We define $f_{drag} = \alpha v$. Now we have the following variables
\begin{align*}
[t_r] &= T \\
[m] &= M \\
[v_0] &= LT^{-1} \\
[g] &= LT^{-2} \\
[\alpha] &= MT^{-1}
\end{align*}
We have $N-r = 2$ dimensionless variables. Define then as
$$\Pi_1 = \frac{gt_r}{v_0} \qquad \Pi_2 = \frac{\alpha v_0}{mg}$$
Note: writing $\Pi_2$ in terms of $t_r$ wouldn't work because by $B\Pi T$, we'd then have
$$\Pi_1 = f\left(\Pi_2\right)$$
would would have $t_r$ in terms of $t_r$. \\
So using our defined dimensionless variables, we have
\begin{align*}
\Pi_1 &= f(\Pi_2) \\
\frac{gt_r}{v_0} &= f\left(\frac{\alpha v_0}{mg}\right) \\
t_r &= \frac{v_0}{g}f\left(\frac{\alpha v_0}{mg}\right)
\end{align*}
\item[(d)] Solve the new DE

\vspace{0.5em}\noindent
By Newton's 2nd law
\begin{align*}
F &= ma \\
ma &= -mg -\alpha v \\
a &= -g - \frac{\alpha v}{m} \\
\frac{dv}{dt} &= -g - \frac{\alpha v}{m} \\
\frac{dv}{dt} + \left(\frac{\alpha}{m}\right)v &= -g
\end{align*}
This DE is in standard form. We solve $v_h(t) = Ce^{-\frac{\alpha}{m}t}$ and $v_p(t) = A \implies v_p(t) = -\frac{gm}{\alpha}$. So our solution to our DE after ICs is
$$v(t) = \left(v_0 + \frac{gm}{\alpha}\right)e^{-\frac{\alpha}{m}t} - \frac{gm}{\alpha}$$
Now we integrate to solve for $h(t)$:
$$h(t) = -\frac{m}{\alpha}\left(v_0 + \frac{gm}{\alpha}\right)e^{-\frac{\alpha}{m}t} - \frac{gm}{\alpha}t + C$$
where $C = \frac{mv_0}{\alpha} + \frac{gm^2}{\alpha^2}$
Solve for $h(t_r) = 0$. To do this, we can apply an approx Taylor expansion to simplify the exponent
$$e^{-\frac{\alpha}{m}t} \approx 1 - \frac{\alpha}{m}t + \frac{\alpha^2}{2m^2}t^2$$
And so we have
$$t_r = \left(\frac{v_0}{g}\right)\left[\frac{2}{\frac{\alpha v_0}{mg} + 1}\right] = \frac{v_0}{g}f\left(\frac{\alpha v_0}{mg}\right) \qquad \qed$$
\end{enumerate}
%END%
\end{document}