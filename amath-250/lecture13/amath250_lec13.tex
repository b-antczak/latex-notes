\documentclass{report}
\usepackage[margin=1in, paperwidth=8.5in, paperheight=11in]{geometry}
%Math packages%
\usepackage{amsmath}
\usepackage{amsthm}
\usepackage{amssymb}
%Spacing%
\usepackage{setspace}
\onehalfspacing
%Lecture number%
\newcommand{\lectureNum}{13}
%Variables - Date and Course%
\newcommand{\curDate}{May 30, 2017}
\newcommand{\course}{AMATH 250}
\newcommand{\instructor}{Zoran Miskovic}
%Defining the example tag%
%\theoremstyle{definition}%
\newtheorem{ex}{Example}[section]
%Setting counter given the lecture number%
\setcounter{chapter}{\lectureNum{}}
\usepackage{graphicx}
\usepackage{calc}
\usepackage{listings}

\lstset{frame=tb,
  aboveskip=3mm,
  belowskip=3mm,
  showstringspaces=false,
  columns=flexible,
  basicstyle={\small\ttfamily},
  numbers=none,
  breakatwhitespace=true,
  tabsize=3
}


\begin{document}
%Note title%
\begin{center}
\begin{Large}
\textsc{\course{} | Lecture \lectureNum{}}
\end{Large}
\end{center} 
\noindent \textit{Bartosz Antczak} \hfill
\textit{Instructor: \instructor{}} \hfill
\textit{\curDate{}}
\rule{\textwidth}{0.4pt}
% Actual Notes%
\subsubsection{Last time}
Finished $B\Pi T$ and started 2nd order DEs with the mechanical oscillation example.
\section{World's Simplest 2nd order DE (3.1.2)}
Here, we neglect friction $\alpha = 0$, and we also consider free oscillation $F(t) = F_{ext} = 0.$ Our ICs are $y(0) = y_0$ and $\frac{dy}{dt}(0) = v_0$ (an initial bump). We see our ``simplified" DE is
\begin{equation}
\frac{d^2y}{dt^2} + \frac{k}{m}y = 0
\end{equation}
Define frequency as
$$\omega = \sqrt{\frac{k}{m}} \implies [\omega] = T^{-1}$$
Let's make our DE non-dimensional. Define
\begin{itemize}
\item Characteristic time: $\frac{1}{\omega}$
\item Dimensionless time: $\tau = \frac{t}{t_c} = \omega t$
\end{itemize}
We now convert our DE to
$$\frac{d^2y}{dt^2} + \omega^2y = 0$$
Let $y(t) = y(\tau(t))$. By Chain Rule
$$\frac{dy}{dt} = \frac{dy}{d\tau}\cdot \frac{d\tau}{dt} = \omega \frac{dy}{d\tau}$$
We need the 2nd derivative as well:
\begin{align}
\frac{d^2y}{dt^2} &= \omega \frac{d}{dt}\left(\frac{dy}{d\tau}\right) \\
&= \omega \frac{d^2y}{d\tau^2}\frac{d\tau}{dt} \\
&= \omega^2 \frac{d^2y}{d\tau^2}
\end{align}
Sub into our DE to get
$$\frac{d^2y}{d\tau^2} + y = 0$$
Recall that 
\begin{align*}
\frac{d^2}{d\tau^2}(\cos \tau) &= -\cos \tau \\
\frac{d^2}{d\tau^2}(\sin \tau) &= -\sin \tau 
\end{align*}
By inspection $y_1(\tau) = \cos \tau$ and $y_1(\tau) = \sin \tau$ are 2 possible solutions to our simplest DE. Thanks to the fact that our DE is linear, we also have
$$y(\tau) = cy_1(\tau) + cy_2(\tau)$$
as the general solution $\forall \; c_1, c_2 \in \mathrm{R}$. The proof is trivial and avoided. \\
Let's solve our IVP with our IVs defined previously:
$$y(t) = c_1 \cos (\omega t) + c_2 \sin (\omega t)$$
\begin{itemize}
\item $y(0) = 0 \implies  c_1 = y_0$
\item $\frac{dy}{dt}(0) = v_0 \implies c_2 = \frac{v_0}{\omega}$
\end{itemize}
Our solution is then
$$y(t) = y_0 \cos (\omega t) + \frac{v_0}{\omega} \sin (\omega t)$$
\subsection{Solving 2nd order DEs (3.2)}
\subsubsection{Fundamental property of linear DEs}
Given the 2nd order linear DE for $y(x)$
\begin{equation}
y^{\prime\prime} + P(x)y^\prime + Q(x)y = 0
\end{equation}
If $y_1(x)$ and $y_2(x)$ are solutions of the DE (13.5), then
$$y(x) = c_1y_1(x) + c_2y_2(x)$$
is a solution of (13.5) $\forall \; c_1,c_2 \in \mathrm{R}$.

\vspace{0.5em}\noindent\textit{Proof:} \\
Let
\begin{align*}
1. \quad c_1y_1^{\prime\prime} + Pc_1y_1^\prime + Qc_1y_1 = 0 \\
2. \quad c_2y_2^{\prime\prime} + Pc_2y_2^\prime + Qc_2y_2 = 0
\end{align*}
We see that we have
\begin{align*}
(c_1y_1^{\prime\prime} + c_2y_2^{\prime\prime}) + P(c_1y_1^{\prime} + c_2y_2^{\prime}) + Q(c_1y_1 + c_2y_2) &= 0 \\
y^{\prime\prime} + Py^\prime + Qy &= 0 \qquad \qed
\end{align*}
%END%
\end{document}