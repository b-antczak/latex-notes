\documentclass{report}
\usepackage[margin=1in, paperwidth=8.5in, paperheight=11in]{geometry}
%Math packages%
\usepackage{amsmath}
\usepackage{amsthm}
\usepackage{amssymb}
%Spacing%
\usepackage{setspace}
\onehalfspacing
%Lecture number%
\newcommand{\lectureNum}{4}
%Variables - Date and Course%
\newcommand{\curDate}{May 9, 2017}
\newcommand{\course}{AMATH 250}
\newcommand{\instructor}{Zoran Miskovic}
%Defining the example tag%
%\theoremstyle{definition}%
\newtheorem{ex}{Example}[section]
%Setting counter given the lecture number%
\setcounter{chapter}{\lectureNum{}}
\usepackage{graphicx}
\usepackage{calc}
\usepackage{listings}

\lstset{frame=tb,
  aboveskip=3mm,
  belowskip=3mm,
  showstringspaces=false,
  columns=flexible,
  basicstyle={\small\ttfamily},
  numbers=none,
  breakatwhitespace=true,
  tabsize=3
}


\begin{document}
%Note title%
\begin{center}
\begin{Large}
\textsc{\course{} | Lecture \lectureNum{}}
\end{Large}
\end{center} 
\noindent \textit{Bartosz Antczak} \hfill
\textit{Instructor: \instructor{}} \hfill
\textit{\curDate{}}
\rule{\textwidth}{0.4pt}
% Actual Notes%
\subsubsection{Last Time}
We looked at sections 1.2.2 -- 1.2.4. All those sections cover methods of solving 1st order DEs and sketching DEs. \\
We learned that the general solution of a 1st order DE has 1 arbitrary constant.
\section{Existence-Uniqueness theorem}
\begin{center}
    \textit{If $\frac{dy}{dx} = f(x,y)$ is of class $C^1$ (has continuous partial derivatives), then the solution curves of the DE do not intersect.}
\end{center}
\section{First order linear DEs with constant coefficient (1.2.5)}
\begin{ex}
Suppose an amount of money $V_0 = 1,000$ is invested at time $t_0 = 0$ in a fund that pays interest at a constant rate of 5\%/year. Assuming that the interest is compounded continuously in time, what is the value of investment $V(t)$ after $t=10$ years?
\end{ex}
\noindent We want to derive a DE for $V(t)$. Let $\Delta V = V(t + \Delta t) - V(t)$. We have:
$$\frac{\Delta V}{V} \approx r \Delta t$$
with $r = $ relative rate $= 5\% / $year.
If we rearrange and let $\Delta t \to 0$, we have:
\begin{align}
    \lim_{\Delta t \to 0} \frac{\Delta V}{\Delta t} &= r V \\
    \frac{dV}{dt} &= rV \\
    \int \frac{dV}{V} &= \int r \; dt\\
    \ln |V| &= rt + c \\
    |V| &= e^ce^{rt} \\
    V &= De^{rt} && \text{(Let }D = \pm e^c)
\end{align}
What happens if I start adding/withdrawing money to/from my account at some rate $f(t)$? We have:
$$\frac{dV}{dt} + kV = f(t)$$
where $k = r$.
\subsubsection{Aside}
The general solution of a DE
\begin{equation}
\frac{dy}{dx} + k(x)y = f(x)    
\end{equation}
has the form $y(x) = y_p(x) + y_h(x)$, where $y_p(x)$ is a particular solution of (4.7), and $y_h(x)$ is a general solution of the associated homogeneous DE
\begin{equation}
\frac{dy}{dx} + k(x)y = 0    
\end{equation}
\subsubsection{Proposition}
If $y_p(x)$ is a particular solution of (4.7) and $y(x)$ is any solution of that same DE, then
$$y_h(x) = y(x) - y_p(x)$$
is a solution of (4.8).

\vspace{0.3cm}\noindent \textit{Proof:} we have
\begin{align*}
    \frac{dy}{dx} + k(x)y = f(x) && \text{(i)}\\
    \frac{dy_p}{dx} + k(x)y_p = f(x) && \text{(ii)}
\end{align*}
Solving (i) - (ii), we have:
$$\frac{d}{dx}(y - y_p) + k(x)[y - y_p] = 0 \implies y_h = y - y_p \qed$$
When $k(x) = k = $ constant, we have
$$\frac{dy_p}{dx} + ky_p = f(x)$$
The general solution is $y(x) = y_h(x) + y_p(x)$. \\
(refer to table on page 17 of textbook).
%END%
\end{document}
