\documentclass{report}
\usepackage[margin=1in, paperwidth=8.5in, paperheight=11in]{geometry}
%Math packages%
\usepackage{amsmath}
\usepackage{amsthm}
\usepackage{amssymb}
%Spacing%
\usepackage{setspace}
\onehalfspacing
%Lecture number%
\newcommand{\lectureNum}{2}
%Variables - Date and Course%
\newcommand{\curDate}{May 4, 2017}
\newcommand{\course}{AMATH 250}
\newcommand{\instructor}{Zoran Miskovic}
%Defining the example tag%
%\theoremstyle{definition}%
\newtheorem{ex}{Example}[section]
%Setting counter given the lecture number%
\setcounter{chapter}{\lectureNum{}}
\usepackage{graphicx}
\usepackage{calc}
\usepackage{listings}

\lstset{frame=tb,
  aboveskip=3mm,
  belowskip=3mm,
  showstringspaces=false,
  columns=flexible,
  basicstyle={\small\ttfamily},
  numbers=none,
  breakatwhitespace=true,
  tabsize=3
}


\begin{document}
%Note title%
\begin{center}
\begin{Large}
\textsc{\course{} | Lecture \lectureNum{}}
\end{Large}
\end{center} 
\noindent \textit{Bartosz Antczak} \hfill
\textit{Instructor: \instructor{}} \hfill
\textit{\curDate{}}
\rule{\textwidth}{0.4pt}
% Actual Notes%
\section{Classification Schemes for DEs}
\begin{enumerate}
    \item \textit{Number of variables}
    \begin{enumerate}
        \item Ordinary DEs
        \item Partial DEs
        \item Systems of DEs
    \end{enumerate}
    (we'll only look at (a) and (c) in this course)
    \item \textit{Order of an ordinary DE for some function $y(x)$}
    \begin{itemize}
        \item The order of an ordinary DE $y(x)$ is the order of the highest derivative of $y(x)$. In general, an $n-$th order DE has the form:
        $$F(x, y, y^\prime, \ldots, y^{(n)}) = 0$$
        or also
        $$y^{(n)} = f(x, y, y^\prime, \ldots, y^{(n-1)})$$
    \end{itemize}
    \item \textit{Linearity}
    \begin{itemize}
    \item 
    If $F$ is a linear function of $y, y^\prime, \ldots, y^{(n)}$, then 
    \begin{equation}
    a_n(x) y^{(n)} + a_{n-1}(x) y^{(n-1)} + \ldots + a_1(x) y^\prime + a_0(x) y = h(x)
    \end{equation}
    where $a_i(x)$ and $h(x)$ are given functions of $x$.
    \end{itemize}
    \item \textit{Homogeneity of a linear DE}
    \begin{itemize}
        \item For (2.1), if $h(x) = 0$ for all $x$, then the equation is a \textbf{homogeneous DE}; otherwise, the DE is non-homogeneous. 
    \end{itemize}
\end{enumerate}
\begin{ex}
Classify the given DEs for $y(x)$
\end{ex}
\begin{enumerate}
    \item[a)] 
$y^\prime = x\sqrt{y}$ \\
\textbf{Solution:} 1st order and non-linear

    \item[b)] $y^{\prime\prime} + 2y^\prime - 3y = 0$ \\ \textbf{Solution:} 2nd order, linear, homogeneous
\end{enumerate}
\subsection{Satisfying a DE}
What does it mean when we say that a function $y(x)$ satisfies a DE? It simply means that it satisfies the equation itself.
\begin{ex}
Show that the given functions satisfy the DEs in the previous example
\end{ex}
\begin{enumerate}
    \item[a)] $y = \frac{x^4}{16}$
    
    \vspace{0.2cm}
    \textbf{Solution:} $\frac{1}{4}x^3 = x \frac{x^2}{4} = \frac{1}{4}x^3 \qed$
    \item[b)] $y_1 = e^x$ and $y_2 = e^{-3x}$ 
    
    \vspace{0.2cm}
    \textbf{Solution:}
    \begin{enumerate}
        \item[1)] it's trivial to see that $y_1 = e^x$ satisfies the DE
        \item[2)] $9e^{-3x} - 6e^{-3x} - 3e^{-3x} = 0 \qed$
    \end{enumerate}
\end{enumerate}
For (b), notice that any function of the form \begin{equation}
    y(x) = c_1e^x + c_2e^{-3x} \qquad c_1, c_2 \in \mathbb{R}
\end{equation}
is also a solution of the DE.

\vspace{0.2cm}\noindent
\textit{Proof:    } Plugging in $y(x) = c_1e^x + c_2e^{-3x}$ into our DE yields:
\begin{align}
    &(c_1e^x + c_2e^{-3x})^{\prime\prime} + 2(c_1e^x + c_2e^{-3x})^\prime - 3(c_1e^x + c_2e^{-3x}) = 0 \\
    &(c_1y_1 + c_2y_2)^{\prime\prime} + 2(c_1y_1 + c_2y_2)^\prime - 3(c_1y_1 + c_2y_2) = 0 \\
    &c_1(y^{\prime\prime} + 2y^\prime - 3y) + c_2(y^{\prime\prime} + 2y^\prime - 3y) = 0 \\
    &0c_1 + 0c_2 = 0 \qed
\end{align}
We refer to (2.2) as the \textbf{general solution} of the DE. The general solution for (a) is $\left(\frac{x^2}{4} + c\right)^2$ for $c \in \mathbb{R}$.
\section{Mathematical Aspects of 1st-order DEs}
The general form of a first order DE is
$$\frac{dy}{dx} = f(x,y)$$
where $y$ represents an \textit{unknown function} $y=y(x)$ (not a variable), and $x$ is an \textit{independent variable}.
\subsection{Separable DEs}
The general form of a first order separable DE is:
$$\frac{dy}{dx} = A(x) B(y)$$
where $A(x)$ and $B(y)$ are arbitrary functions. The general approach to solving these equations is shown:
\begin{align*}
    \frac{1}{B(y)}\cdot \frac{dy}{dx} &= A(x) && \text{(Divide by } B(y) \neq 0) \\
    \int \frac{1}{B(y)}\cdot \frac{dy}{dx} \; dx &= \int A(x) \; dx \\
    \int \frac{1}{B(y)} \; dy &= \int A(x) \; dx
\end{align*}
\begin{ex}
Solve $\frac{dy}{dx} = x\sqrt{y}$
\begin{align*}
    \frac{dy}{\sqrt{y}} &= x \cdot dx \\
    \int \frac{1}{\sqrt{y}} \; dy &= \int x \; dx \\
    2 \sqrt{y} + c_1 &= \frac{1}{2} x^2 + c_2 \\
    y &= \left(\frac{1}{4}x^2 + c \right) \qed 
\end{align*}
\end{ex}
\begin{ex}
Solve $\frac{dy}{dx} = -\frac{x}{y}$
\begin{align*}
    \frac{dy}{y} &= -x \cdot dx \\
    \int \frac{1}{y} \; dy &= - \int x \; dx \\
    x^2 + y^2 &= c \qed
\end{align*}
This is an implicit solution.
\end{ex}
%END%
\end{document}
