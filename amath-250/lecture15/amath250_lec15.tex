\documentclass{report}
\usepackage[margin=1in, paperwidth=8.5in, paperheight=11in]{geometry}
%Math packages%
\usepackage{amsmath}
\usepackage{amsthm}
\usepackage{amssymb}
%Spacing%
\usepackage{setspace}
\onehalfspacing
%Lecture number%
\newcommand{\lectureNum}{15}
%Variables - Date and Course%
\newcommand{\curDate}{June 4, 2017}
\newcommand{\course}{AMATH 250}
\newcommand{\instructor}{Zoran Miskovic}
%Defining the example tag%
%\theoremstyle{definition}%
\newtheorem{ex}{Example}[section]
%Setting counter given the lecture number%
\setcounter{chapter}{\lectureNum{}}
\usepackage{graphicx}
\usepackage{calc}
\usepackage{listings}

\lstset{frame=tb,
  aboveskip=3mm,
  belowskip=3mm,
  showstringspaces=false,
  columns=flexible,
  basicstyle={\small\ttfamily},
  numbers=none,
  breakatwhitespace=true,
  tabsize=3
}


\begin{document}
%Note title%
\begin{center}
\begin{Large}
\textsc{\course{} | Lecture \lectureNum{}}
\end{Large}
\end{center} 
\noindent \textit{Bartosz Antczak} \hfill
\textit{Instructor: \instructor{}} \hfill
\textit{\curDate{}}
\rule{\textwidth}{0.4pt}
% Actual Notes%
\subsubsection{Last time}
General solution of DE with constant coefficient for $y(x)$
\begin{equation}
y^{\prime\prime} + py^\prime + qy = 0
\end{equation}
where $p, q$ are constants. Our characteristic equation is
$$\lambda^2 + p\lambda + q = 0$$
Let the two solutions to that formula as $\lambda_1$ and $\lambda_2$. Let
\begin{align*}
y_1(x) &= e^{\lambda_1x} \\
y_2(x) &= e^{\lambda_2x}
\end{align*}
From this, our solution to (15.1) is
$$y(x) = c_1y_1(x) + c_2y_2(x)$$
We have three cases for our constants
\subsubsection{Case 1}
$$p^2 > 4q \implies \lambda_1 \neq \lambda_2$$
\subsubsection{Case 2}
$$p^2 < 4q \implies \lambda_1 = a + ib, \quad \lambda_2 = a - ib$$
Here, our solutions are
\begin{align*}
y_1(x) &= e^{ax}\cos (bx) \\
y_2(x) &= e^{ax}\sin (bx)
\end{align*}
\begin{ex}
Find general solution for $y^{\prime\prime} + 4y^\prime + 5y$
\end{ex}\noindent
Our characteristic equation is
$$\lambda^2 +4\lambda + 5 = 0$$
with solutions $\lambda_1 = -2 + i$ and $\lambda_2 = -2 - i$. With here, we let $a = -2$ and $b=1$ to yield a solution
$$y(x) = c_1e^{-2x}\cos x + c_2e^{-2x}\sin x \qquad \qed$$
\subsubsection{Case 3}
$$p^2 = 4q \implies \lambda_1 = \lambda_2 = \lambda = -\frac{p}{2}$$
Here we do something fancy. We define our trial function as
$$y(x) = U(x)e^{\lambda x}$$ With these, we have
\begin{align*}
y^\prime &= U^\prime e^{\lambda x} + U\lambda e^{\lambda x} \\
y^{\prime\prime} &= U^{\prime\prime}e^{\lambda x} + 2U^\prime \lambda e^{\lambda x} + U\lambda^2e^{\lambda x}
\end{align*}
Plug into our general 2nd order DE (15.1)
\begin{align}
\left[U^{\prime\prime} + 2\lambda U^\prime + U\lambda^2 + p(U^\prime + U\lambda) + qU\right]e^{\lambda x} &= 0 \\
\left[U(\lambda^2 + \lambda p + q) + U^\prime(2\lambda + p) + U^{\prime\prime}\right]e^{\lambda x} &= 0 \\
\implies U^{\prime\prime}(x) &= 0
\end{align}
From this, our trial function is
$$y(x) = (c_1x + c_2)e^{\lambda x}$$
If one solution of (15.1) is $y_2(x) = e^{\lambda_2 x}$ and $\lambda_1 = \lambda_2$, then multiply $y_2(x)$ by a factor of $x$ to get second linear independent solution.
\begin{ex}
Solve $y^{\prime\prime} -2y^\prime + y = 0$
\end{ex}\noindent
The solution to our characteristic equation is $\lambda_2 = \lambda_2 = 1$
So $y_2(x) = e^x$, and using the result from last time $y_2(x) = xe^x$, so our given solution is
$$y(x) = c_1e^x + c_2xe^x \qquad \qed$$
\subsection{General solution of non-homogeneous DE with constant coefficients (3.2.4)}
Our general formula is
\begin{equation}
y^{\prime\prime} + py^\prime + qy = f(x)
\end{equation}
where $f(x)$ is generally given of the form:
\begin{itemize}
\item Polynomial in terms of $x$
\item Exponent of $x$
\item sin or cos function
\item The product or sum of any combination of the previous functions
\end{itemize} $y(x)$ is unknown. Recall the general solution is
$$y(x) = y_h(x) + y_p(x)$$
where $y_h(x)$ is the general solution of the associated homogeneous 2nd degree linear DE, and $y_p(x)$ is the particular solution of (15.5). We use the method of undetermined coefficients to find $y_p(x)$.\newpage
\begin{ex}
Solve the IVP: $y^{\prime\prime} - y^\prime -2y = 4x$ with $y(0) = 2$ and $y^\prime(0) = 1$
\end{ex}\noindent
There are 12 formal steps we need to follow on exams:
\begin{enumerate}
\item Define associated homogeneous DE
$$y^{\prime\prime} - y^\prime -2y = 0$$
\item Define characteristic equation for $$\lambda^2 - \lambda - 2 = 0$$
\item Solve
$$\lambda_1 = 2, \qquad \lambda_2 = -1$$
\item Find general solution of associated homogeneous DE
$$y_h(x) = c_1e^{2x} + c_2e^{-x}$$
\item Find $y_p(x)$. To do this, make a trial solution. We determine this by looking at our function $f(x)$. Assume $y_p(x) = A + Bx$.
\item Solve for $A$ and $B$:
$$A = 1, \qquad B=-2$$
\item Some other simple step that I didn't bother writing down but wanted to maintain order number integrity
\item Build our particular solution
$$y_p(x) = 1-2x$$
\item General solution of DE is
$$y(x) = c_1e^{2x} + c_2e^{-x} + 1 - 2x$$
\item Impose the ICs
\begin{align*}
y(0) &= c_1 + c_2 + 1 = 2 \\
y^\prime(0) &= 2c_1 - c_2 - 2 = 1
\end{align*}
\item Solve system:
$$c_1 = \frac{4}{3}, \qquad c_2 = -\frac{1}{3}$$
\item Plug back into general solution, and we're done
$$y(x) = \frac{4}{3}e^{2x} - \frac{1}{3}e^{-x} + 1 - 2x \qquad \qed$$
\end{enumerate}
%END%
\end{document}