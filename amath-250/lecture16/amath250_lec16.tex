\documentclass{report}
\usepackage[margin=1in, paperwidth=8.5in, paperheight=11in]{geometry}
%Math packages%
\usepackage{amsmath}
\usepackage{amsthm}
\usepackage{amssymb}
%Spacing%
\usepackage{setspace}
\onehalfspacing
%Lecture number%
\newcommand{\lectureNum}{16}
%Variables - Date and Course%
\newcommand{\curDate}{June 6, 2017}
\newcommand{\course}{AMATH 250}
\newcommand{\instructor}{Zoran Miskovic}
%Defining the example tag%
%\theoremstyle{definition}%
\newtheorem{ex}{Example}[section]
%Setting counter given the lecture number%
\setcounter{chapter}{\lectureNum{}}
\usepackage{graphicx}
\usepackage{calc}
\usepackage{listings}

\lstset{frame=tb,
  aboveskip=3mm,
  belowskip=3mm,
  showstringspaces=false,
  columns=flexible,
  basicstyle={\small\ttfamily},
  numbers=none,
  breakatwhitespace=true,
  tabsize=3
}


\begin{document}
%Note title%
\begin{center}
\begin{Large}
\textsc{\course{} | Lecture \lectureNum{}}
\end{Large}
\end{center} 
\noindent \textit{Bartosz Antczak} \hfill
\textit{Instructor: \instructor{}} \hfill
\textit{\curDate{}}
\rule{\textwidth}{0.4pt}
% Actual Notes%
\subsubsection{Last time}
Method of undetermined coefficients for non-homogeneous DEs. We discussed the 12 steps$^{\textsc{tm}}$ to solving it. Today we'll look at more examples of it.
\begin{ex}
Find $y_p(x)$ for $y^{\prime\prime} - y^\prime - 2y = \sin x$
\end{ex}\noindent
Assume $y_p(x) = A\cos(x) + B\sin(x)$.
\begin{align*}
y_p^\prime (x) &= -A\sin (x) + B\cos(x) \\
y_p^{\prime\prime} (x)&= -A\cos(x) - B\sin(x)
\end{align*}
Subbing into our DE and solving for $A,B$, we get
$$y_p(x)= \frac{1}{10}\cos(x) - \frac{3}{10}\sin(x) \quad \qed$$
\begin{ex}
Find $y_p(x)$ for $y^{\prime\prime} - y^\prime - 2y = 4x + e^x$
\end{ex}\noindent
Let $f(x) = f_1(x) + f_2(x)$ where
\begin{align*}
f_1(x) &= 4x \\
f_2(x) &= e^x
\end{align*}
Due to linearity, we can write $y_p(x) = y_{p1}(x) + y_{p2}(x)$ where
$$y_{pi}^{\prime\prime} + y_{pi}^\prime - 2y_{pi} = f_i(x) \qquad i = 1,2$$
We know from last lecture that $y_{p1}(x) = 1-2x$. For $y_{p2}(x) = De^x = y_{p2}^\prime(x) = y_{p2}^{\prime\prime}(x)$, we solve for $D = -\frac{1}{2}$, and so our solution is
$$y_p(x) = 1-2x - \frac{e^x}{2} \qquad \qed$$
\begin{ex}
Find $y_p(x)$ for $y^{\prime\prime} - y^\prime - 2y = 4xe^x$
\end{ex}\noindent
Assume $y_p(x) = (Ax + B)e^x$
\begin{align*}
y_p^\prime(x) &= (Ax + A + B)e^x \\
y_p^{\prime\prime}(x) &= (Ax + 2A + B)e^x
\end{align*}
Sub into our DE and solve for $A$ and $B$, and we get $A = -2$ and $B = -1$
$$y_p(x) = (-2x-1)e^x$$
\newpage
\begin{ex}
Find $y_p(x)$ for $y^{\prime\prime} - y^\prime - 2y = 4xe^{-x}$
\end{ex}\noindent
We cannot assume $y_p(x) = (Ax + B)e^{-x}$ because this will yield our LHS $= -3A = 4x$, which is wrong because $A$ is a coefficient, not a function! $Be^{-x}$ repeats the term $c_2e^{-x}$ in $y_h(x)$.\\
We must discuss some exceptions to the method of undetermined coefficients:
\begin{center}
\textit{If any term in the assumed $y_p(x)$ repeats a term in $y_h(x)$ , then multiply your $y_p(x)$ by a factor of $x$}
\end{center}
So we assume $y_p(x) = (Ax^2 + Bx)e^{-x}$
\begin{align*}
y_p^\prime(x) &= (-Ax^2 + 2Ax - Bx + B)e^{-x} \\
y_p^{\prime\prime}(x) &= (Ax^2 -4Ax + Bx + 2A - 2B)e^{-x}
\end{align*}
Solving, we get $A = -\frac{2}{3}$ and $B = -\frac{4}{9}$ and so
$$y_p(x) = -\left(\frac{2}{3}x^2 + \frac{4}{9}x\right)e^{-x} \qquad \qed$$
\begin{ex}
Height of baseball thrown upward
\end{ex}
$$\frac{d^2h}{dt^2} = -g$$
Our ICs are $h(0) = 0, h^\prime(0) = v_0$. Using the 12 magic$^\textsc{tm}$ steps,
\begin{enumerate}
\item $h^{\prime\prime}(t) = 0$
\item $\lambda^2 = 0$
\item $h_1(t) = e^0 = 1$, $h_2 = t$
\item $h_h(t) = c_1 + c_2t$
\item $h_p(t) = t^2A$ (since $tA$ is already in $h_h(t)$)
\item $h^{\prime\prime}_p(t) = -g = 2A \implies A = -\frac{1}{2}g$
\item $h(t) = c_1 + c_2t - \frac{1}{2}gt^2 \quad \qed$
\end{enumerate}

%END%
\end{document}