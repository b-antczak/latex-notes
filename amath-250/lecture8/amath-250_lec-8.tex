\documentclass{report}
\usepackage[margin=1in, paperwidth=8.5in, paperheight=11in]{geometry}
%Math packages%
\usepackage{amsmath}
\usepackage{amsthm}
\usepackage{amssymb}
%Spacing%
\usepackage{setspace}
\onehalfspacing
%Lecture number%
\newcommand{\lectureNum}{8}
%Variables - Date and Course%
\newcommand{\curDate}{May 18, 2017}
\newcommand{\course}{AMATH 250}
\newcommand{\instructor}{Zoran Miskovic}
%Defining the example tag%
%\theoremstyle{definition}%
\newtheorem{ex}{Example}[section]
%Setting counter given the lecture number%
\setcounter{chapter}{\lectureNum{}}
\usepackage{graphicx}
\usepackage{calc}
\usepackage{listings}

\lstset{frame=tb,
  aboveskip=3mm,
  belowskip=3mm,
  showstringspaces=false,
  columns=flexible,
  basicstyle={\small\ttfamily},
  numbers=none,
  breakatwhitespace=true,
  tabsize=3
}


\begin{document}
%Note title%
\begin{center}
\begin{Large}
\textsc{\course{} | Lecture \lectureNum{}}
\end{Large}
\end{center} 
\noindent \textit{Bartosz Antczak} \hfill
\textit{Instructor: \instructor{}} \hfill
\textit{\curDate{}}
\rule{\textwidth}{0.4pt}
% Actual Notes%
\subsubsection{Last Time}
Section 1.3 in the notes.
\section{Chapter 2: Dimensional Analysis}
\subsection{Characteristic scales and dimensionless variables}
\begin{ex}
Continuously compounded interest at 5\% / year
\end{ex}\noindent
Recall that
$$\frac{dV}{dt} = rV$$
with $V(t) = $ value of investment, $V(0) = 0$, and $r$ is our interest rate of 5\%/year. We have $[r] = T^{-1}$. We define the \textit{characteristic time} as
$$t_c = \frac{1}{r} \implies [t_c] = T$$
What does this characteristic time represent? Recall the solution to our DE:
$$V(t) = V_0e^{rt} \implies V(t_c) = V_0e$$
This means that $t_c$ is the time it takes investment to increase by a factor of $e$. \\
We now define \textit{dimensionless time} as a new independent variable:
$$\tau = \frac{t}{t_c} = rt \implies [\tau] = 1$$
We want to change our DE:
$$V(t) = \widetilde{V}(\tau) = \widetilde{V}(\tau(t))$$
where $\tau(t) = rt$. We use chain rule:
$$\frac{dV}{dt} = \frac{d\widetilde{V}}{d\tau} \cdot \frac{d\tau}{dt} = \frac{d\widetilde{V}}{d\tau} r$$
Substitute into our DE:
$$\frac{d\widetilde{V}}{d\tau}r = r\widetilde{V}$$
$$\implies \frac{d\widetilde{V}}{d\tau} = \widetilde{V}$$
We often abuse notation: $V(t) = V(\tau)$.
\begin{ex}
Exponential decay of radioactive substance
\end{ex}\noindent
Carbon dating uses radioactive isotope $^{14}C$ with half-life $t_{\frac{1}{2}} = 5,730$ years. Let $m(t)$ be the amount (mass) of $^{14}C$ at time $t \geq 0$ with $m(0)=m_0$. By definition, $m\left(t_{\frac{1}{2}}\right) = \frac{1}{2}m_0$. Our DE for radioactive decay is
$$\frac{dm}{dt} = -km$$
where $k > 0$, a constant. We know that the solution to this DE is
$$m(t) = m_0e^{-kt}$$
Let's express $k$ in terms of $t_{\frac{1}{2}}$:
\begin{align}
m\left(t_\frac{1}{2}\right) &= m_0e^{-kt_{\frac{1}{2}}} = \frac{1}{2}m_0\\
e^{-kt_{\frac{1}{2}}} &= \frac{1}{2} \\
-kt_{\frac{1}{2}} &= \ln 2 \\
k &= \frac{\ln 2}{t_{\frac{1}{2}}}
\end{align}
So we see that
$$\frac{dm}{dt} = -\frac{\ln 2}{t_{\frac{1}{2}}}m$$
Let's write it in dimensionless form
$$t_c = \frac{1}{k}$$
$$\tau = \frac{t}{t_c} = kt = \ln 2 \frac{t}{t_{\frac{1}{2}}}$$
If we let $m(t) = m(\tau), \tau = kt$, then we use the chain rule to solve
\begin{align}
\frac{dm}{dt} = \frac{dm}{d\tau}\frac{d\tau}{dt} = k\frac{dm}{d\tau} \\
\implies \frac{dm}{d\tau} = -m
\end{align}
\begin{ex}
Motion of a baseball thrown vertically up from surface of Earth with initial velocity $v_0$
\end{ex}\noindent
We see that $h(0) = 0$ and $h^\prime(0) = v_0$. By Newton's 2nd law, we have
\begin{align}
m\frac{d^2h}{dt^2} &= -mg \\
\frac{d^2h}{dt^2} &= -g
\end{align}
Now we simply integrate
\begin{align*}
\frac{dh}{dt}  = \int \frac{d^2h}{dt^2} \; dt = -gt + c_1 \\
h(t) =g\frac{t^2}{2} + c_1t + c_2
\end{align*}
Using our ICs, we see that $c_2 = 0$ and $c_1 = v_0$.
%END%
\end{document}