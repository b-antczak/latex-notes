\documentclass{report}
\usepackage[margin=1in, paperwidth=8.5in, paperheight=11in]{geometry}
%Math packages%
\usepackage{amsmath}
\usepackage{amssymb}
%\usepackage{MnSymbol}
\usepackage{amsthm}
%Spacing%
\usepackage{setspace}
\onehalfspacing
%Lecture number%
\newcommand{\lectureNum}{24}
%Variables - Date and Course%
\newcommand{\curDate}{March 6, 2017}
\newcommand{\course}{MATH 239}
\newcommand{\instructor}{Luke Postle}
%Defining the example tag%
%\theoremstyle{definition}%
\newtheorem{ex}{Example}[section]
%Setting counter given the lecture number%
\setcounter{chapter}{\lectureNum{}}
%Package for drawing graphs%
\usepackage{tikz}
\usepackage{verbatim}
\usetikzlibrary{arrows}

\begin{document}
%Note title%
\begin{center}
\begin{Large}
\textsc{\course{} | Lecture \lectureNum{}}
\end{Large}
\end{center} 
\noindent \textit{Bartosz Antczak} \hfill
\textit{Instructor: \instructor{}} \hfill
\textit{\curDate{}}
\rule{\textwidth}{0.4pt}
%Actual Notes%
\subsubsection{Review of last lecture}
A \textbf{permutation} of $[n] = \{1,2,\cdots, n\}$ is a \textit{sequence} (i.e., ordered) of distinct elements of $[n]$. The number of permutations of $[n]$ of length $k$ is $$\frac{n!}{(n-k)!}$$
A \textbf{combination} of $[n]$ is a set (i.e., unordered) of distinct elements of $[n]$. The number of combinations of $n$ of size $k$ is ``$n$ choose $k$", denoted $${n \choose k} = \frac{n!}{(n-k)!k!}$$
\begin{ex}
Some combinations
\end{ex}
\begin{align*}
n \choose 0 &= 1 \\
n \choose 1 &= n \\
n \choose n-1 &= n \\
n \choose n &= 1
\end{align*}
\section{Identities and Proving Combinatorially}
Recall that the binomial theorem is
$$(1 + x)^n = \sum_{k=0}^n {n \choose k}x^k$$
Let's prove this combinatorially:
\subsubsection{Proof of Binomial Theorem}
The coefficient of $x^k$ in the expansion of $(1+x)^n$ is equal to the number of terms which have chosen $x$ $k$ times from the product $(1 + x^n)$ But this is equivalent to choosing a $k$-element subset of $[n]$, i.e., $n \choose k$.
\subsection{Definition | Combinatorial Identity}
A \textbf{combinatorial identity} is an equation which relates combinatorial objects/numbers.
\begin{ex}
Combinatorial identity \#1
\end{ex}
$${n \choose k} = {n \choose n-k}$$
\subsubsection{Combinatorial Proof of identity 1}
Let $S_{n,k}$ be the subsets of $[n]$ of size $k$. Recall that (by definition):
$$|S_{n,k}| = {n \choose k}$$
Let $S_{n,n-k}$ be the subsets of $[n]$ of size $k$. So,
$$|S_{n, n-k}| = {n \choose n-k}$$
Now all we need to do is show that these two sets are of equal size. We do this using \textit{bijection}.\\
We claim that $S_{n,k}$ is in bijection with $S_{n, n-k}$ (here, $S$ is a subset of $[n]$ of size $k$):
$$f: S_{n,k} \rightarrow S_{n,n-k}: f(S) = [n] - S$$
Now let's consider the inverse (here, $T$ is the subset of $[n]$ of size $n-k$):
$$f^{-1}:S_{n,n-k} \rightarrow S_{n,k}:f^{-1}(T) = [n]-T$$
Thus, $|S_{n,k}| = |S_{n,n-k}|$, ergo $${n \choose k} = {n \choose n-k}$$.
\begin{ex}
Combinatorial identity \#2
\end{ex}
$${n \choose k} = {n-1 \choose k-1} = {n-1 \choose k}$$
\subsubsection{Combinatorial Proof of identity 2}
Let $S_{n,k}$ denote the subsets of $[n]$ of size $k$.\\Let $A_1$ be the subsets of $[n]$ of size $k$ that contain $n$ (i.e., the last element in $[n]$). Let $A_2$ be the subsets of $[n]$ of size $k$ that contain $n$. Thus,
$$S_{n,k} = A_1 \sqcup A_2 \qquad \text{(The disjoint union)}$$ Hence $$|S_{n,k}| = |A_1| + |A_2| = {n \choose k}$$
What is the size of $A_1$? We claim that $A_1$ is in bijection with $S_{n-1, k-1}$. Let's write the bijection (Here, $S \in A_1$):
$$f: A_1 \rightarrow S_{n-1,k-1} : f(S) = S - \{n\}$$
The inverse is (Here, $T \in S_{n-1,k-1}$):
$$f^{-1}: S_{n-1,k-1} \rightarrow A_1 : f^{-1}(T) = T \cup \{n\}$$
Thus, $|A_1| = |S_{n-1,k-1}|$\\
What is the size of $A_2$ We claim that $A_1$ is in bijection with $S_{n-1,k}$. Let's write the bijection:
$$f: A_2 \rightarrow S_{n-1,k}: f(S) = S$$
The inverse is:
$$f: S_{n-1,k} \rightarrow A_2 :  f(T) = T$$
Thus, $|A_2| = |S_{n-1,k}|$.\\
So,
$${n \choose k} = |S_{n,k}| = |A_1| + |A_2| = {n-1 \choose k} + {n-1 \choose k}$$
\subsubsection{Final note about combinations}
\begin{itemize}
\item \textbf{Q:} how many subsets of $[n]$ are there of size $k$?
$$n \choose k$$
\item \textbf{Q:} how many binary strings are there of length $n$ with $k$ 1's?
$$n \choose k$$
\item \textbf{Q:} how many binary strings are there of length $n$ with $k$ 0's (or $n-k$ 1's)?
$${n \choose k} = {n \choose n-k}$$
\item \textbf{Q:} how many compositions are there of $n$ with $k$ parts? (we're to give a different bijection of this on the homework)
$${n-1 \choose k-1}$$
\end{itemize}
Now, let's stop counting combinatorially and actually \textit{count algebraically}.
\section{An Algebraic View of Counting}
What if we wanted to find the number of a certain type of object (e.g., the number of binary string of size $n > 5$ that don't contain the substring ``101").\\What if we think of that type we're looking far as an \textit{unknown variable}. Usually we don't want to count one type (e.g., binary strings), but many types (e.g., binary strings of length $n$).\\
Let's \textbf{encode} these infinitely many variables as coefficients of an infinite polynomial. We'll focus on this topic for the next two weeks.
%END%
\end{document}