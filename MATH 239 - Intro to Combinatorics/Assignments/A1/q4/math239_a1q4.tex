\documentclass{report}
\usepackage[margin=1in, paperwidth=8.5in, paperheight=11in]{geometry}
%Math packages%
\usepackage{amsmath}
\usepackage{amsthm}
%Spacing%
\usepackage{setspace}

\begin{document}
\noindent \textbf{MATH 239 | Assignment 1, Question 4}\\
Bartosz Antczak (ID: 20603468)
\onehalfspacing
\begin{enumerate}
\item[a)] Recall the handshaking lemma
$$\sum_{v\in V} \mathrm{deg}(v) = 2\vert E \vert$$
Also recall that the sum of an even amount of odd numbers is always even; whereas the sum of an odd amount of odd numbers is always odd.
Since every vertex in $G$ is odd, we must always have an even number of vertices in order for the sum to be even, QED.
\item[b)] By a corollary of the handshaking lemma, the average degree of the vertices of a graph is
$$\frac{\displaystyle\sum_{v \,\in\, V} \mathrm{deg}(v)}{\vert V \vert} = \frac{2 \vert E \vert}{\vert V \vert}$$
Since $\vert E \vert = \vert V \vert$, we have
$$\frac{\displaystyle\sum_{v \,\in\, V} \mathrm{deg}(v)}{\vert V \vert} = 2$$
Since the average degree of the vertices is 2 where the only two possible degrees are 1 and 3, this must mean that there are an equal number of vertices of degree 1 and degree 3.
\end{enumerate}
\end{document}