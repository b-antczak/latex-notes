\documentclass{report}
\usepackage[margin=1in, paperwidth=8.5in, paperheight=11in]{geometry}
%Math packages%
\usepackage{amsmath}
\usepackage{amsthm}
%Spacing%
\usepackage{setspace}
\onehalfspacing
%Lecture number%
\newcommand{\lectureNum}{6}
%Variables - Date and Course%
\newcommand{\curDate}{February 15, 2017}
\newcommand{\course}{MATH 239}
\newcommand{\instructor}{}
%Defining the example tag%
%\theoremstyle{definition}%
\newtheorem{ex}{Example}[section]
%Setting counter given the lecture number%
\setcounter{chapter}{\lectureNum{}}
%Package for drawing graphs%
\usepackage{tikz}
\usepackage{verbatim}
\usetikzlibrary{arrows}

\begin{document}
%Note title%
\begin{center}
\begin{Large}
\textsc{\course{} | Tutorial \lectureNum{}}
\end{Large}
\end{center} 
\noindent \textit{Bartosz Antczak} \hfill
\textit{\curDate{}}
\rule{\textwidth}{0.4pt}
% Actual Notes%
\section*{Matchings}
Recall that a \textbf{matching} is a set of edges that share no end points. A \textbf{vertex cover} is a set $C$ of vertices such that every edge has at least one end in the set. For all $G$, the max matching is less than or equal to the min vertex cover.
\subsection*{Algorithm for max matching in bipartite graphs}
\begin{enumerate}
\item Begin with any matching $M$
\item Construct $X$ and $Y$:
\begin{enumerate}
\item $X_0$ is the set of vertices in $A$ that are unsaturated by $M$
\item $Z$ is the set of vertices reachable from $X_0$ by an alternating path
\item $X = A \cap Z$, and $Y = B \cap Z$
\end{enumerate}
\item If there's an unsaturated $v \in Y$, find an augmenting path $P$ ending at $v$; use it to construct a larger matching $M^\prime$. Replace $M$ by $M^\prime$ and go to step 2.
\item If every vertex is saturated, then stop. $M$ is a max matching.
\end{enumerate}

\section*{Example: Problem Set 8.3 - Q5}
On our first iteration, we have:
\begin{enumerate}
\item $X_0 = \{1,2\}$
\item $Z = \{1,2,3,4,5, a,b,c,d,e\}$ (i.e., every vertex), so $A \cap Z = A = X$ and $B \cap Z = B = Y$ are  
\item 
\end{enumerate}
%END%

\end{document}