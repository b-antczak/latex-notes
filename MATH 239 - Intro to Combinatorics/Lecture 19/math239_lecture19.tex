\documentclass{report}
\usepackage[margin=1in, paperwidth=8.5in, paperheight=11in]{geometry}
%Math packages%
\usepackage{amsmath}
\usepackage{amssymb}
\usepackage{amsthm}
%Spacing%
\usepackage{setspace}
\onehalfspacing
%Lecture number%
\newcommand{\lectureNum}{19}
%Variables - Date and Course%
\newcommand{\curDate}{February 15, 2017}
\newcommand{\course}{MATH 239}
\newcommand{\instructor}{Luke Postle}
%Defining the example tag%
%\theoremstyle{definition}%
\newtheorem{ex}{Example}[section]
%Setting counter given the lecture number%
\setcounter{chapter}{\lectureNum{}}
%Package for drawing graphs%
\usepackage{tikz}
\usepackage{verbatim}
\usetikzlibrary{arrows}

\begin{document}
%Note title%
\begin{center}
\begin{Large}
\textsc{\course{} | Lecture \lectureNum{}}
\end{Large}
\end{center} 
\noindent \textit{Bartosz Antczak} \hfill
\textit{Instructor: \instructor{}} \hfill
\textit{\curDate{}}
\rule{\textwidth}{0.4pt}
%Actual Notes%
\section{More on Matchings}
We asked a question last lecture: when is the size of a max matching equal to the size of a min cover? Today we will analyse that:
\begin{itemize}
\item For $K_n$, the max matching is $\lfloor \frac{n}{2}\rfloor$ and the min cover is $n-1$
\item For $C_{2k+1}$ (odd cycle), the max matching is $k$, and the min cover is $k+1$. If there isn't an odd cycle, we are bipartite and 2-colourable.
\item For $k$ triangles (disconnected triangles), the max matching is $k$, and the min cover is $2k$
\end{itemize}
Since the two values are not equal in an odd cycle, what if we considered graphs without one? Specifically a bipartite graph. Could they be equal for a bipartite graph? From this stems \textbf{Konig's Theorem}.
\subsection{Konig's Theorem}
\begin{center}
\textit{If G is bipartite, then the size of the max matching of G is equal to the size of the min cover}
\end{center}
\subsubsection{Proof of Konig's Theorem}
\textit{(We will construct a matching and cover of equal size)}\\
Let $A$ and $B$ be the two bipartitions of $G$. Let $M$ be a maximum matching. Also, let $X_0$ be the unsaturated vertices of $A$ ($X_0$ is non-empty as otherwise $M$ has size $|A|$ and yet $A$ is a cover of size $|A|$, as desired). Let $Z$ be the set of all vertices that are the end of an alternating path whose other end is in $X_0$.\\ Let $X = A \cap Z$, and $Y = B \cap Z$. Let $C = (A - X) \cup Y$. I claim that $C$ is the cover of the same size as $M$, which I will prove.
\begin{itemize}
\item \textbf{Claim:} there is no edge with one end in $X$ and the other end in $B-Y$.
\subsubsection{Proof of Claim}
Let $uv, u \in X$, and $v \in B-Y$, be such an edge. Since $u \in X$, there exists an alternating path $P$ from a vertex of $X_0$ to $u$. Either $u \in X_0$ and $P$ is one vertex or the last edge is in $M$. If $u \in X_0$, then $uv$ is an alternating path from $u$ to $v$ and so $v \in Y$, a contradiction.\\
If $u \not\in X_0$, then either $v$ is not matched to $u$ in $M$ and then $P^\prime = P + uv$ is an alternating path from a vertex of $X_0$ to $v$, and so $v \in Y$, a contradiction. Otherwise, $v$ is matched in $M$ to $u$, but then $v$ is on $P$ already and so $P^\prime = P - uv$ is an alternating path and $v \in Y$, a contradiction.\\
It follows from the claim that every edge has an end in at least one of $A-X$ or $Y$. Thus, $C$ is a cover!
\item \textbf{Claim 2:} every vertex of $A-X$ is saturated.
\subsubsection{Proof of Claim 2}
If there exists an unsaturated vertex, then it would be in $X_0$ and so in $X$.
\item \textbf{Claim 3:} every vertex of $Y$ is saturated.
\subsubsection{Proof of Claim 3}
If there existed a vertex which was unsaturated, then there would exist an alternating path with one end in $X_0$ and othe other end would be unsaturated in $Y$. But then $P$ is an augmenting path and so $M$ is not maximum, a contradiction.
\end{itemize}
From these claims, we see that every vertex in $C$ is saturated.
\begin{itemize}
\item \textbf{Claim 4:} there exists an edge $e = uv \in M$ such that $u \in Y$, and $v \in A-X$.
\subsubsection{Proof of Claim 4}
Since $u \in Y$, there exists an alternating path $P$ from a vertex of $X_0$ to $u$. The last edge of $P$ mus not be in $M$, by parity. But then $P^\prime = P+uv$ is an alternating path from that vertex of $X_0$ to $v$. So $v \in X$, a contradiction.
\end{itemize}
Finally, we claim that $|C| = |E(M)|$, and so $C$ is a min cover and $M$ is a max matching, by our earlier lemma.\\
To see this, let $M_1$ be the set of edges of $M$ with an end in $Y$ and let $M_2$ be the set of edges of $M$ with an end in $A-X$. By our last claim, these are disjoint sets. Since $C$ is a cover, every edge of $M$ is in either $M_1$ or $M_2$. Since every vertex of $Y$ is saturated, $|Y| = |E(M_1)|$, and similarly, since every vertex of $A-X$ is saturated, $|A-X| = |E(M_2)|$. So,
$$|E(M)| = |E(M_2)| + |E(M_2)| = |Y| + |A-X| = |C|$$
As required (phew!).
\subsection{Algorithm for Finding a Max Matching in a Bipartition}
\begin{enumerate}
\item Start with a matching $M$.
\item Construct $X,Y,C$ from $X_0$.
\item If there exists an unsaturated vertex in $Y$, then there exists an augmenting path $P$ of $M$. Flip $P$ on $M$ and return to step 2.
\item Otherwise, $M$ is a max and $C$ is a min cover.
\end{enumerate}
\subsection{Algorithmic Questions}
For a fixed $k$,
\begin{enumerate}
\item Does $G$ have a matching of size $\geq k$?
\begin{itemize}
\item $NP$? Yes for all graphs.
\item co$-NP$? Yes for \underline{bipartite} graphs. Simply show a cover of size $k-1$.
\end{itemize}
\item Let $M$ be a matching of $G$. Is $M$ maximum?
\begin{itemize}
\item $NP?$ Yes for \underline{bipartite}. Show a cover of the same size.
\item co$-NP$? Yes for all graphs. Simply show a larger matching or an augmenting path.
\end{itemize}
\end{enumerate}

%END%
\end{document}