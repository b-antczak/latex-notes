\documentclass{report}
\usepackage[margin=1in, paperwidth=8.5in, paperheight=11in]{geometry}
%Math packages%
\usepackage{amsmath}
\usepackage{amsthm}
%Spacing%
\usepackage{setspace}
\onehalfspacing
%Lecture number%
\newcommand{\lectureNum}{8}
%Variables - Date and Course%
\newcommand{\curDate}{March 8, 2017}
\newcommand{\course}{MATH 239}
\newcommand{\instructor}{}
%Defining the example tag%
%\theoremstyle{definition}%
\newtheorem{ex}{Example}[section]
%Setting counter given the lecture number%
\setcounter{chapter}{\lectureNum{}}
%Package for drawing graphs%
\usepackage{tikz}
\usepackage{verbatim}
\usetikzlibrary{arrows}

\begin{document}
%Note title%
\begin{center}
\begin{Large}
\textsc{\course{} | Tutorial \lectureNum{}}
\end{Large}
\end{center} 
\noindent \textit{Bartosz Antczak} \hfill
\textit{\curDate{}}
\rule{\textwidth}{0.4pt}
% Actual Notes%
\subsubsection{Recall Binomial Theorem}
$$(1+x)^n = \sum_{k=0}^n {n \choose k}x^k$$
\subsubsection{Generating Series}
Suppose $S$ is a set of configurations and for each $\sigma \in S$, we have some weight $w(\sigma)$ (must be non-negative and an integer). For a given $k$, how many elements of $S$ have weight $k$?\\
We define the \textbf{generating series} of $S$ with respect to $w$ as:
$$\Phi_{S}(x) = \sum_{\sigma \in S}x^{w(\sigma)} = \sum_{k \geq 0} a_kx^k$$
Here, $a_k$ is the number of elements of $S$ with weight $k$.
\begin{ex}
Find $\Phi_S(x)$ if:
\end{ex}
\begin{enumerate}
\item[a)] $S = \{1,2, \cdots\}$ and $w(i) = i$:
\begin{align*}
\Phi_S(x) &= 0x^0 + 1x + 1x^2 + \cdots \\
&= \sum_{i=0}^\infty x^i
\end{align*}
\item[b)] $S = \{1,2, \cdots\}$ and $w(i) = i$ if $i$ is even, $w(i) = i-1$ if $i$ is odd. Let's first construct a table:
\begin{center}
\begin{tabular}{ c | c }
$i$ & $w(i)$ \\\hline
1 & 0 \\
2 & 2 \\
3 & 2 \\
4 & 4 \\
5 & 4 \\
6 & 6 \\
7 & 6 \\
\end{tabular}
\end{center}
From here, our generating series is:
\begin{align*}
\Phi_{s}(x) &= 1x^0 + 0x^1 + 2x^2 + 0x^3 + 2x^4 + \cdots \\
&= 1 + \sum_{i=1}^\infty 2x^{(2i)}
\end{align*}
\end{enumerate}\newpage
\subsubsection{Definition}
$[x^n]A(x)$ is notation that defines the coefficient of $x^n$ in $A(x)$.\\
Let $A(x) = a_0 + a_1x + a_2x^2 + a_3x^3 + \cdots$ and $A(y) = a_0 + a_1y + a_2y^2 + a_3y^3 + \cdots$. We say that $A(x) = A(y)$ if all of the coefficients are the same (which means that we don't care about the variables $x$ and $y$).
\begin{ex}
\end{ex}
Prove that $(1-x)^{-k} = \displaystyle\sum_{n \geq 0} {n+k-1 \choose k-1}x^n$
\subsubsection{Solution}
Use induction.
\begin{itemize}
\item \textbf{Base Case:} let $k=1$. It's clear that this statement holds
\item \textbf{Hypothesis:} assume that this statement holds for some $m$. We'll try to show it works for $m+1$
\item \textbf{Conclusion:} consider $m = m+1$. We have
$$(1-x)^{-(m+1)} = (1-x)^{-m}(1-x)^{-1}$$
By our hypothesis,
$$[x^i](1-x)^{-m} = {i - m - 1 \choose m-1}$$
Also, since
$$\frac{1}{1-x} = 1 + x + x^2 + x^3 + \cdots$$
We see that $[x^i](1-x)^{-1} = 1$.\\
So we have,
\begin{align*}
[x^n](1-x)^{-(m+1)} &= \sum_{i=0}^n ([x^i](1-x)^{-m}[x^{n-i}](1-x)^{-1})\\
&= \sum_{i=0}^n {i + m - 1 \choose m - 1}\\
&= {n + m \choose m} \qquad \text{(Using the solution to problem 1.5.3 in course notes)}
\end{align*}
From here, we see that every coefficient in $(1-x)^{-(m+1)}$ is equal to ${n+m \choose m}$, thus the entire polynomial is equal to:
$$(1-x)^{-(m+1)} = \sum_{n \geq 0} {m+n \choose m} x^n \qquad \qed$$
\begin{center}
\textit{(Here, we have proven the negative binomial series)}
\end{center}\newpage
\end{itemize}
\begin{ex}
\end{ex}
Solve $[x^8](1-x)^{-7}$.
\subsubsection{Solution}
Observe that $[x^8](1-x)^{-7} = [x^8]\displaystyle\sum_{n \geq 0} {n+7-1 \choose 7-1}x^n$. So the coefficient's value when $n=8$ is:
$${8+7-1 \choose 7-1} = {14 \choose 6}$$
\begin{ex}
\end{ex}
Solve $[x^{10}]x^6(1-2x)^{-5}$.
\subsubsection{Solution}
This is equivalent to solving for $[x^4](1-2x)^{-5}$. Now, let $y = 2x$. We have:
\begin{align*}
&= \left[\left(\frac{y}{2}\right)^4\right](1-y)^{-5} \\
&= 2^4[y^4](1-y)^{-5}
\end{align*}
We now have our problem outlined in a way that we can solve it:
$$2^4{4+5-1 \choose 5-1}$$
%END%

\end{document}