\documentclass{report}
\usepackage[margin=1in, paperwidth=8.5in, paperheight=11in]{geometry}
%Math packages%
\usepackage{amsmath}
\usepackage{amssymb}
%\usepackage{MnSymbol}
\usepackage{amsthm}
%Spacing%
\usepackage{setspace}
\onehalfspacing
%Lecture number%
\newcommand{\lectureNum}{30}
%Variables - Date and Course%
\newcommand{\curDate}{March 22, 2017}
\newcommand{\course}{MATH 239}
\newcommand{\instructor}{Luke Postle}
%Defining the example tag%
%\theoremstyle{definition}%
\newtheorem{ex}{Example}[section]
%Setting counter given the lecture number%
\setcounter{chapter}{\lectureNum{}}
%Package for drawing graphs%
\usepackage{tikz}
\usepackage{verbatim}
\usetikzlibrary{arrows}

\begin{document}
%Note title%
\begin{center}
\begin{Large}
\textsc{\course{} | Lecture \lectureNum{}}
\end{Large}
\end{center} 
\noindent \textit{Bartosz Antczak} \hfill
\textit{Instructor: \instructor{}} \hfill
\textit{\curDate{}}
\rule{\textwidth}{0.4pt}
%Actual Notes%
\section{More on Binary Strings}
Let's return to our problem we outlined last lecture:
\begin{center}
How many binary strings are there of length $n$ with no 3 consecutive ones?
\end{center}
To better tackle this problem, we create a definition.
\subsubsection{Definition | substring}
We say $b$ is a \textbf{substring} of $a$ if $\exists \, c, d $ such that $a = cbd$.\\
From this definition, we can restructure our question as such:
\begin{center}
How many strings of length $n$ have no 111 substring?
\end{center}
If we let $D$ be the binary strings with no 111 substring, our goal is to find $\Phi_D(x)$.
\subsection{Solving our Problem}
Let $B$ be the set of all binary strings. We ask ourselves, what is an unambiguous expression for $B$?
\begin{itemize}
\item The trivial solution is $\{0,1\}^*$ (here, we decompose after every 0 or 1)
\item Another approach is to only decompose after every 0. We'll stick with this one
\end{itemize}
Under this decomposition, one piece can look like:
$$0, 10, 110, 1110, \cdots$$
In other words, some number of 1s followed by 0. A more compact way to write this piece is $P = \{1\}^*\{0\}$. However, we can have \textit{any} number of pieces:
$$B = (\varepsilon \cup P \cup P^2 \cup P^3 \cup \cdots )\{1\}^* = P^*\{1\}^* = {(\{1\}^*\{0\})}^*\{1\}^*$$
(Here, we concatenated $\{1\}^*$ at the end to ensure we have matched to every binary string). We state that $B$ is unambiguous. Why? Because it came from a decomposition rule that \textit{uniquely} decomposes every string.\\
Now for $D$, what is the equivalent expression? Under our decomposition, what can one piece in $D$ look like?
\begin{itemize}
\item Beginning/middle piece: $0, 10, 110$
\item End piece: $\varepsilon, 1, 11$
\end{itemize}
For $D$, our expression is:
$D = (\{0, 10, 110\})^*\{\varepsilon, 1, 11\}$
and this is unambiguous, since it is just a restriction of an unambiguous expression for $B$.\\
\subsubsection{Finishing Off}
So,
\begin{align*}
\Phi_D(x) &= \Phi_{\{0,10,110\}^*} \cdot \Phi_{\{\varepsilon, 1, 11\}} (x) & \text{(By Product Lemma)} \\
&= \frac{1}{1 - \Phi_{\{0,10,110\}^*}} \cdot (1 + x + x^2) & \text{(By our Star Lemma)} \\
&= \frac{1 + x + x^2}{1 - (x + x^2 + x^3)} & \qed
\end{align*}
From here, we can solve for the number of binary strings of length $n$ by solving $[x^n]\Phi_D(x)$.
\subsection{Other Decomposition Rules}
\begin{itemize}
\item \textbf{Decompose after every 1} (i.e., symmetric version to decomposing after every 0): $B = {(\{0\}^*\{1\})}^*\{0\}^*$ is useful for solving problems regarding the numbers of 0s in a string
\end{itemize}
We have a more general decomposition rule. Before we state it, let's create a definition:
\subsubsection{Definition | Block}
A \textbf{block} of a binary string is a maximal substring all having the same digit (i.e., all 0s or all 1s). For example, in $00111011$, our blocks are: $00, 111, 0, 11$.
\begin{itemize}
\item \textbf{Block Decomposition Rule}:\\
Decompose after every block of 0s:
\begin{itemize}
\item beginning piece: $\{0\}^*$
\item middle piece: $\{1\}^*\{1\}\{0\}^*\{0\}$
\item end piece: $\{1\}^*$
\end{itemize}
Then, $B =$ beginning $\cdot$ (middle)$^* \cdot$ end $= \{0\}^*{\left(\{1\}^*\{1\}\{0\}^*\{0\}\right)}^*\{1\}^*$
\end{itemize}
\subsection{Problems from Course Notes}
\begin{enumerate}
\item[\textbf{2.7.1.}] \textit{Find the generating series for binary strings with no block of exactly length 2.}
\subsubsection{Solution}
To solve this, let us restrict the block rule to the condition: 	``no block of length exactly two".
\begin{itemize}
\item Beginning piece: $\{\varepsilon, 0, 000, 0000, \cdots\} = \varepsilon \cup \{0\} \cup \{000\}\{0\}^*$
\item Middle piece: $(\{1\} \cup \{111\}\{1\}^*)(\{0\} \cup \{000\}\{0\}^*)$
\item End piece: $\varepsilon \cup \{1\} \cup \{111\}\{1\}^*$
\end{itemize}
Then, our expression is:
$$(\varepsilon \cup \{0\} \cup \{000\}\{0\}^*){((\{1\} \cup \{111\}\{1\}^*)(\{0\} \cup \{000\}\{0\}^*))}^*(\varepsilon \cup \{1\} \cup \{111\}\{1\}^*)$$
\end{enumerate}
To find the geometric series for each piece:
\begin{align*}
&\Phi_{begin}(x) = 1 + x + x^3\cdot \frac{1}{1-x} \\
&\Phi_{mid}(x) = x + x^3\cdot \left(\frac{1}{1-x}\right)\left(x + x^3\cdot \frac{1}{1-x}\right) \\
&\Phi_{end}(x) = 1 + x + x^3 \cdot \frac{1}{1-x}
\end{align*}
So our generating series is:
\begin{align*}
&= \left(1+x+\frac{x^3}{1-x}\right){\left({\left(x+\frac{x^3}{1-x}\right)}^2\right)}^*\left(1 + x + \frac{x^3}{1-x}\right) \\
&= \left(1+x+\frac{x^3}{1-x}\right)\left(\frac{1}{1-(x+\frac{x^3}{1-x})^2}\right)\left(1 + x + \frac{x^3}{1-x}\right)\\
&= \frac{1-x^2+x^3}{1-2x+x^2-x^3} & \qed
\end{align*}
%END%
\end{document}